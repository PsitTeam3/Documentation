\section{Resultat}\label{resultat}

\subsection{Erreichte Ziele}\label{erreichteZiele}

Die Ziele, welche wir uns für die Umsetzung des Projekts im Rahmen des PSIT3 Moduls gesetzt haben, wurden alle erreicht:
\begin{itemize}
  \item Eine Liste mit allen verfügbaren Touren kann angezeigt werden.
  \item Eine Tour kann von der Liste gewählt und gestartet werden.
  \item Die App sendet an das Backend die Koordinaten der Reisenden. Das Backend ist mit dieser Information in der Lage, eine Anfrage an die Google-Maps API zu stellen und die Route zu einem Besichtigungspunkt der Tour zurückzugeben.
  \item Die Kartenfunktionalität wurde erfolgreich implementiert. Die Karte zeigt die Route zum nächsten Ziel. Dabei wird die Route für die Fortbewegung zu Fuss angezeigt. Dem Kunden wird keine andere Möglichkeit angezeigt, wie er an das Ziel gelangen kann. Ursprünglich war hier die Idee, dass der Touristin verschiedene Optionen angezeigt werden, unter anderem die Integration von Uber.
  \item Während dem sich die Reisende fortbewegt, wird die Route periodisch angepasst.
  \item Die App erkennt, wenn der Reisende sich in der Nähe des Ziels befindet und es werden sofort die Informationen zu dem Besichtigungspunkt eingeblendet.
  \item An einem Besichtigungspunkt angekommen, wird der Tourist ebenfalls aufgefordert, ein Foto eines bestimmten Objektes zu schiessen. All diese verschiedenen Fotos werden lokal gespeichert und später für die Zusammenfassung der Tour verwendet.
  \item Am Ende der Tour wird dem Touristen eine Zusammenfassung seiner Tour angezeigt, welche unter anderem die persönlichen Fotos der Tour enthält.
\end{itemize}

Die Travelbuddy API, das Backend-System, erfüllt alle funktionalen Anforderungen, welche durch den Umfang der Android-App gegeben sind. Das beudetet, dass die Android-App bereits alle benötigten Daten von der API beziehen kann. Das Backend ist auch bereits an einen SQL-Server angebunden, um die Daten zu persistieren.


\subsection{Offene Punkte}\label{offenePunkte}
Einige Punkte der Projektidee standen nicht im Fokus und sind darum noch offen:

\begin{itemize}
  \item Die Fotos welche die Anwenderin während der Tour macht, werden nur lokal auf dem Gerät gespeichert. Die Fotos sollten aber an das Backend gesendet werden, damit die Fotos persistiert werden, aber auch von einem anderen Gerät her abgefragt werden könnten.
  \item Die Benutzerauthentifizierung fehlt noch komplett. Die App funktioniert aktuell nur für einen Benutzer. Das Backend ist jedoch schon für die Verwendung mehrerer Anwender vorbereitet. Über die API ist es möglich, Touren für verschiedene Benutzerinnen zu starten. Dazu werden für die verschiedenen Aufrufe jeweils eine Benutzer-ID als Parameter mitgegeben. Es fehlt jedoch auch noch die Authentifizierung auf Backend-Seite, da noch keine Passwörter vorgesehen sind, bzw. kein Passwort-Hash zusammen mit der Benutzer-ID mitgegeben werden kann.
  \item Touren können nicht gefiltert werden, es wird lediglich eine Liste mit allen verfügbaren Touren angezeigt.
  \item Die personalisierte Zusammenfassung der Tour bei Tourbeendung kann nicht auf einem Social-Media Account geteilt werden.
  \item Eine gestartete Tour kann nicht unterbrochen oder abgebrochen werden. Wenn der Tourist die App schliest, kann die Tour nicht fortgesetzt werden.
\end{itemize}


\subsection{Weiterentwicklungsmöglichkeiten}\label{weiterentwicklungsmöglichkeiten}
Nachfolgend einige Ideen, wie das Travelbuddy-Projekt fortgesetzt werden könnte:
\begin{itemize}
  \item Zur Berechnung der Route fragt die Android-App das Backend an, die Route vom aktuellen Standort der Touristin zum nächsten Punkt zurückzugeben. Das Backend wiederum macht daraufhin eine Anfrage an Google-Maps. Da dies periodisch in kurzen Zeitintervallen geschieht, ist dies nicht besondern performant. Besser wäre es, wenn der Client die Route gleich selbst bei Google-Maps abfragt.
  \item Die Integration von Uber war in der Projektidee enthalten, wurde aber in der Anforderungsanalyse verworfen. Dies wäre aber ein für die Touristen hilfreiches Feature.
  \item Grosses Potential sehen wir im Bezug auf Augmented Reality. Es wäre genial, wenn man während einer Tour auf dem Gerät Informationen über interessante Objekte angezeigt bekäme, welche gerade im Sichtfeld des Anwenders sind. Also wenn die Anwenderin z.B. eine Tour durch das Niederdorf in Zürich macht und in der Nähe des Grossmünsters ist, würde auf dem Gerät das Grossmünster abgebildet werden und gleich dazu Informationen über die Kirche und allenfalls auch gleiches für den Brunnen daneben und das Café gegenüber. Man könnte das Konzept auch sehr gut für Museen verwenden, wo man unter Verwendung einer Holo-Lens digital geführte Touren durch das Museum oder eine Ausstellung anbieten könnte. Mit einer Holo-Lens könnte eine Applikation geschrieben werden, welche die verschiedenen Gegenstände erkennen und Informationen in beliebiger Form in den Raum projezieren kann. Um diese Idee umzusetzen würde am besten ein platformunabhängiges Framework entwickelt werden. Darauf aufbauend könnten verschiedene Client-Applikationen für unterschiedliche Geräte und auch unterschiedliche Zwecke (Museums-Tour, Städte-Tour, Gebäude-Tour für grosse Neubauten und vieles mehr) entwickelt werden.
  \item Eine Web-Applikation wäre hilfreich für die Tourvorbereitung, sowie für das Ansehen der Tour-Zusammenfassungen. Es könnte also zusätzlich eine Web-Oberfläche umgesetzt werden, mit welcher der Tourist an seinem Computer Zuhause alle verfügbaren Touren für seine Feriendestination durchgehen und diejenigen speichern könnte, welche er schlussendlich durchführen möchte. Diese Arbeit ist produktiver an einem Laptop oder Computer als auf einem Smartphone oder Tablet. Zudem könnte die Oberfläche ebenfalls so gestalltet werden, dass die Touristin nach Beendigung einer Tour dort ihre persönliche Zusammenfassung inlusive ihrer Fotos anschauen und auch auf einem Social-Media Profil teilen könnte.
  \item Touren können momentan nur direkt via SQL erfasst werden. Das ist unpraktikabel. Sämtliche Daten müssten mit Hilfe einer Web-Oberfläche erfasst werden können. Dazu müsste eine eigene Web-Applikation entwickelt werden, die dann auch von Reisebüros verwendet würde.
\end{itemize}
